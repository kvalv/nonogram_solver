\subsection*{choice of \texttt{h} function}
A state consists of candidates for each row and column. Let $nc_i$ be the
number of column candidates for column $i$. Similarly, we define $nr_j$ as the
number of candidate rows for row $j$. Then, $h$ is defined as

\begin{displaymath}
  h = \sum\limits_{i \in \texttt{cols}} log(nc_i) + \sum\limits_{j \in \texttt{rows}} log(nr_j)
\end{displaymath}

Logarithm's are used to prevent numerical overflow. This could occur if there
are a large number of rows / columns or if the range is sufficiently large. If any
$nc_i = 1$ then they will not contribute to the heuristic function because $log(1) = 0$.

\subsection*{Choice of variable on next assumption}
The choice of variable on which to base the next assumption is solely based
on the number of candidates (domain cardinality) for that variable. If there
are few candidates, then it will be faster to visit all possible states by
assuming specific values for that variable. On the other hand, if the domain
is large, it'll take more time to visit all subsequent nodes. Therefore, the
line with fewest candidates (greater than 1) is chosen. The algorithm below
implements this and returns the line index with smallest domain.

\begin{algorithm}
\begin{footnotesize}
\begin{lstlisting}[language=Python, caption=Algorithm used to find the row / column with smallest domain.]
def fewest_candidates(lines):
  lengths = np.array(
    [len(each) for each in lines])

  lengths[lengths == 1] = max(lengths)

  return np.argsort(lengths)[0]
\end{lstlisting}
\end{footnotesize}
\end{algorithm}

This heuristic is important because if there are a lot of candidates for a specific line,
then it will be computationally expensive to generate assumptions on each one of them.
Since only one candidate will be correct for each line, we would like to reduce the number
of assumptions made. That is the main reason for implementing the \texttt{fewest\_candidates}
algorithm to reduce the number of assumptions made. This made larger puzzles, such as
the \textit{reindeer}-puzzle computationally feasible.
