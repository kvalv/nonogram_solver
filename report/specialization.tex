Several tweaks of the \texttt{A*}-algorithm had to be done in order for the
nonogram solver to work. The \texttt{A*}-algorithm is implemented as an abstract base class in Python.
This means that several methods had to be implemented.  The methods that had to
be implemented were as follows:

\begin{itemize}
    \item \textbf{\texttt{goal\_fun}}: Specifies if a certain state has reached
    the goal or not. Done by checking if $\forall_i: \|\mathcal{D}(x_i)\| = 1$
    holds or not.

    \item \textbf{\texttt{cost\_fun}}: Specifies the cost of going from \texttt{parent}
    to \texttt{child} where \texttt{parent} and \texttt{child} are
    Node-instances. In this case, the cost function returns a 1. The function
    could also be 0, but this makes it more susceptible to deep nesting of
    states.
    
    \item \textbf{\texttt{generate\_children}}: This is the heart of the
    \texttt{A*}-algorithm. Specifies how the children are generated. Here also
    the domain creation + constraint enforcement is done.

\end{itemize}
