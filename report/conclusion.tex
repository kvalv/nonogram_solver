A nonogram solver using GAC and \texttt{A*} was implemented. A good representation of the
state was important to get good performance. A direct representation of the puzzle
is possible, albeit it is not effective to apply in computations. Therefore, an aggregate representation was used.

Tweaks had to be done to improve the performance. Most notably was the use of heuristic functions which made some 
problems feasible, and improved the solving times by a minimum factor of 7.

The solver could improve its performance by implementing more elaborate heuristics. The article
\textit{Solving Japanese Puzzles with Heuristics} by Salcedo-Sanz et. al gives some pointers. 
In addition, the \texttt{copy.deepcopy} method could be changed to reduce time spent on copying data objects.   




